\documentclass{article}
\usepackage{fullpage}
\usepackage{amsmath, amssymb, amsfonts}
\usepackage{graphicx}

\begin{document}

\title{EMSR-b and DAVN Heuristics in Airline Revenue Management}
\author{}
\date{}
\maketitle

\section{Introduction}

Airline revenue management revolves around the challenge of determining which 
customer bookings to accept and which to reject, aiming to maximize overall 
revenue. Airlines offer multiple fare classes for the same origin-destination 
itinerary, each associated with different conditions and privileges. As tickets 
are sold over time, airlines must strategically adjust fare availability, 
closing lower fare classes and opening higher ones to optimize profitability.

Passengers in the economy cabin typically have access to several fare classes, 
even though all seats in that cabin are physically identical. Lower fare classes 
tend to sell out earlier due to higher demand, while higher fare classes may 
remain available longer, often offering additional benefits such as flexible 
cancellations or priority boarding. While customers who book early often choose 
lower fares, some may still opt for premium fares despite cheaper options being 
available. Airlines frequently adjust pricing based on factors such as remaining 
time before departure, demand patterns, and customer preferences to enhance 
revenue.

The key decision in revenue management is how many seats to allocate to each 
fare class to strike a balance between high occupancy and maximizing revenue. 
Allocating too many seats to lower fares may lead to full flights but lower 
profits, while reserving too many seats for higher fares risks unsold inventory, 
reducing potential revenue. Thus, airlines must carefully manage seat 
allocations to avoid these extremes.

Additionally, airline seats are a perishable resource—once a flight departs, any 
unsold seat represents lost revenue. To mitigate this, airlines employ 
overbooking strategies, selling more tickets than available seats to compensate 
for expected cancellations and no-shows. However, if more passengers show up 
than seats available, the airline incurs costs by compensating displaced 
travelers and arranging alternative flights. The complexity of revenue 
management arises from fluctuating demand, cancellation rates, and overbooking 
risks, all of which require airlines to make data-driven decisions on pricing 
and seat allocation.

The Expected Marginal Seat Revenue (EMSR-b) heuristic is widely used to 
determine seat allocations among different fare classes dynamically. It extends 
Littlewood’s rule by incorporating multiple fare classes and probabilistic 
demand forecasts. The Displacement Adjusted Virtual Nesting (DAVN) heuristic 
builds upon EMSR-b to handle multi-leg airline networks, taking into account how 
bookings on one leg impact availability across the network, thereby optimizing 
revenue across multiple connected flights.

\section{Definitions of Key Concepts}
Before delving into the EMSR-b and DAVN heuristics, we define some key technical 
terms:
\begin{itemize}
    \item \textbf{Fare Class} ($i$): A category of airline tickets distinguished 
by price and associated conditions (e.g., refundable vs. non-refundable).
    \item \textbf{Fare Price} ($f_i$): The price of fare class $i$.
    \item \textbf{Protection Level} ($Q_i$): The number of seats reserved for 
fare classes higher than and including~$i$ to ensure that they are available for 
potential high-revenue passengers. There is no protection level for the class 
with lowest fare.
    \item \textbf{Booking Limit} ($B_i$): The maximum number of seats that can 
be sold at a specific fare class $i$.
    \item \textbf{Demand} ($D_i$):
    % \item \textbf{Demand Distribution}: The statistical representation of 
% expected customer demand for each fare class, typically modeled using a normal 
% distribution.
    % \item \textbf{Revenue Management}: The practice of dynamically adjusting 
% seat allocations to maximize revenue while considering demand uncertainty.
    % \item \textbf{Displacement Cost}: The opportunity cost of accepting a 
% booking, which might prevent a more profitable booking later.
\end{itemize}

\section{EMSR-b Heuristic: Booking limits for a single leg}

Littlewood’s rule provides the foundation for airline revenue management by 
addressing the trade-off between selling a seat at a lower fare now versus 
reserving it for a potentially higher-paying customer in the future. The key 
idea is that an airline should accept a booking at a lower fare only if the 
expected revenue from selling it now is at least as great as the expected 
revenue from saving it for a later, higher-fare customer.

Mathematically, in a two-fare class system, Littlewood’s rule states that a seat 
should be sold at fare $f_1$ if
\[
    f_1 \geq f_2\cdot \Pr[D_2 > Q_2],
\]
where $\Pr[D_2 > Q_2]$ represents the probability that demand $D_2$ for the 
higher fare $f_2$ exceeds the protection level~$Q_2$. If this probability is 
high, the airline should protect more seats for high-fare customers. If it is 
low, the airline can safely sell more seats at the lower fare.

The EMSR-b heuristic generalizes Littlewood’s rule to multiple fare classes. 
Instead of considering only two fares, EMSR-b calculates a cumulative demand 
distribution for all fare classes above a given class and determines seat 
protection levels accordingly. The core idea remains the same: balance the 
trade-off between immediate revenue and potential higher future revenue.

By considering aggregated demand distributions and adjusting booking limits 
dynamically, EMSR-b enables airlines to make optimal seat allocation decisions 
that maximize total expected revenue.

The process consists of the following steps to compute the booking limits of the 
fare classes. Other data, such as airplane capacity, fare prices, demands, and 
demand distributions, are assumed to be part of the input.

\subsubsection*{Step 1: Sorting Fare Classes}
Sort all fare classes in ascending order by fare price, such that
\[
    f_1 \leq f_2 \leq \dots \leq f_n.
\]

\subsubsection*{Step 2: Compute Aggregated Demand and Fares}
For each fare class $i$, define the aggregated demand for all fare classes above 
and including $i$:
\[
    D_{\geq i} = \sum_{j=i}^{n} D_j,
\]
and the aggregate fares as
\[
    f_{\geq i} =  \frac{\sum_{j=i}^{n} f_j \cdot E[D_j]}{\sum_{j=i}^{n} E[D_j]}.
\]
% Assume demand follows a normal distribution with mean and variance:
% \[
%     \mu_{\geq i} = \sum_{j=i}^{n} \mu_j, \quad \sigma_{\geq i}^2 = 
% \sum_{j=i}^{n} \sigma_j^2.
% \]

\subsubsection*{Step 3: Compute Protection Levels}
The lowest fare class $1$ has no protection level as no protection is needed for 
the lowest fare.
For the remaining fare classes, determine the protection levels by solving 
Littlewood’s equation for $Q_{i+1}$:
\[
    f_i = f_{\geq i+1} \cdot \Pr[D_{i+1} > Q_{i+1}].
\]
A common assumption is that the arrivals are Poisson distributed, which means 
the inverse cumulative distribution function of the Poisson distribution can be 
used to solve for protection level~$Q_{i+1}$ in Littlewood's equation.
% where $z_i$ is the critical fractile computed as $z_i = \Phi^{-1} \left( 
% \frac{f_i}{f_n} \right)$,
% and $\Phi^{-1}$ is the inverse of the cumulative distribution function of the 
% normal distribution.

\subsubsection*{Step 4: Compute Booking Limits}
The booking limit for each fare class $i<n$ is given by:
\[
    B_i = C - Q_{i+1}.
\]
For the highest fare class, $B_n = C$, where $C$ is the capacity of the 
airplane.

The booking limit essentially means that once $B_i$ seats have been reserved in 
class $i$, no additional bookings are permitted for that class. When overbooking 
is taken into account, $C$ is replaced with $C / (1 + cp)$ above, where $cp$ 
represents the average cancellation probability across all fare classes. This 
modification temporarily increases the available capacity during the booking 
process to account for expected cancellations.

\section{The DAVN Heuristic: Extending EMSR-b to Networks}
While EMSR-b optimizes seat allocations for a single flight leg, the DAVN 
heuristic extends this concept to multi-leg itineraries.

\subsection{Linear Program Formulation}
The DAVN heuristic solves the following linear program:
\[
    \max \sum_{j=1}^{n} f_j x_j
\]
subject to:
\begin{align*}
    0 \leq x_j \leq E[D_j], \quad &\forall j = 1, 2, ..., n\\
    \sum_{j \in A_\ell} x_j \leq C_\ell,\quad &\forall \ell = 1, 2, ..., L 
\qquad\qquad (\star)
\end{align*}
where $f_j$ is the fare price of product $j$, $E[D_j]$ is the expected demand 
for product $j$, $C_\ell$ is the capacity of leg~$\ell$, and $A_\ell$ is the set 
of products using leg~$\ell$.

\subsection{Computing Displacement Adjusted Revenue (DARE)}
The displacement adjusted revenue (DARE) for product $j$ on leg $\ell$ is 
computed as:
\[
    DARE_j^\ell = f_j - \sum_{i \neq \ell} \lambda_i,
\]
where $\lambda_\ell$ is the dual price of constraint $(\star)$ for leg $\ell$. 
These DARE values are then used as fares to apply EMSR-b on each leg separately, 
in order to determine the booking limits for each leg.


\end{document}
